\PassOptionsToPackage{dvipsnames}{xcolor}
\documentclass[10pt,a4paper,ragged2e,withhyper]{altacv}
\geometry{left=1.2cm,right=1.2cm,top=1cm,bottom=1cm,columnsep=0.75cm}
\usepackage{paracol}


\ifxetexorluatex
  \setmainfont{Roboto Slab}
  \setsansfont{Lato}
  \renewcommand{\familydefault}{\sfdefault}
\else
  \usepackage[rm]{roboto}
  \usepackage[defaultsans]{lato}
  \renewcommand{\familydefault}{\sfdefault}
\fi

  \definecolor{PrimaryColor}{HTML}{0b486b}
  \definecolor{SecondaryColor}{HTML}{0b486b}
  \definecolor{ThirdColor}{HTML}{0b486b}
  \definecolor{BodyColor}{HTML}{666666}
  \definecolor{EmphasisColor}{HTML}{2E2E2E}
  \definecolor{BackgroundColor}{HTML}{ffffff}


\colorlet{name}{PrimaryColor}
\colorlet{tagline}{PrimaryColor}
\colorlet{heading}{PrimaryColor}
\colorlet{headingrule}{ThirdColor}
\colorlet{subheading}{SecondaryColor}
\colorlet{accent}{SecondaryColor}
\colorlet{emphasis}{EmphasisColor}
\colorlet{body}{BodyColor}
\pagecolor{BackgroundColor}

\renewcommand{\namefont}{\Huge\rmfamily\bfseries}
\renewcommand{\personalinfofont}{\small\bfseries}
\renewcommand{\cvsectionfont}{\LARGE\rmfamily\bfseries}
\renewcommand{\cvsubsectionfont}{\large\bfseries}

\renewcommand{\itemmarker}{{\small\textbullet}}
\renewcommand{\ratingmarker}{\faCircle}

\begin{document}
    \name{Benjamin MILHET}
    \tagline{Etudiant Ingénieur en Informatique \\
    Ingéniérie du Logiciel et des Connaissances}
    %% You can add multiple photos on the left or right
    \photoL{4cm}{qrcode}
    
    \personalinfo{
        \email{benjamin.milhet@outlook.fr}\smallskip
        \phone{+33 6 83 70 09 22}
        \NewInfoField{location}{\faMapMarker}[https://www.google.fr/maps/place/21000+Dijon/@47.3318644,4.9971998,13z/data=!3m1!4b1!4m5!3m4!1s0x47f29d8ceffd9675:0x409ce34b31458d0!8m2!3d47.322047!4d5.04148]
        \location{Dijon, France}\\
        \NewInfoField{calendar}{\faCalendar}
        \calendar{19/10/2000}
        %\linkedin{Benjamin MILHET}
        \github{Orchanyne}
    }
    
    \makecvheader
    
    %% Set the left/right column width ratio to 6:4.
    \columnratio{0.25}
    \begin{paracol}{2}
    
         % ----- ABOUT ME -----
        \cvsection{Profil}
            \begin{quote}
                Je suis actuellement en 1ère année du cursus ingénieur en Informatique / Électronique et je suis à la recherche d'un stage d'un mois pour se familiariser avec le monde du travail.
            \end{quote}
            
            \bigskip
        % ----- ABOUT ME -----

        % ----- SKILLS -----
        \cvsection{Compétences}
            \smallskip
            \cvtag{Java}
            \cvtag{C++}
            \cvtag{Python}
            \cvtag{C}
            \cvtag{C\#}
            \cvtag{Javascript}
            \cvtag{PHP}
            \cvtag{SQL}
            \medskip
            
            \cvtag{Unity}
            \cvtag{Symfony}
            \cvtag{Node.js}
            \cvtag{Git}
            \cvtag{Shell}
            \cvtag{LaTeX}
            \bigskip
        % ----- SKILLS -----
        
        % ----- LANGUAGES -----
        \cvsection{Langues}
        \smallskip
            \cvlang{Anglais}{Courant / B2}\\
            \divider
            \cvlang{Espagnol}{Base / A2}
        \bigskip
        \smallskip
        % ----- LANGUAGES -----
        
        % ----- INTERESTS -----
        \cvsection{Passions}
        \smallskip
            \cvevent{}{\cvrepo{ \faCode ~ Raspberry pi}{}}{}{}
            \divider
            
            \cvevent{}{\cvrepo{ \faGamepad ~ Jeux vidéo}{}}{}{}
            \divider
            
            \cvevent{}{\cvrepo{ \faBookOpen ~ Comics}{}}{}{}
            \divider
            
            \cvevent{}{\cvrepo{ \faBicycle ~ VTT}{}}{}{}

        % ----- INTERESTS -----
        %\begin{figure}[!h]
        %    \centering
        %    \includegraphics[width=0.15\textwidth]{qrcode.png}
        %\end{figure}
        \newpage
        \switchcolumn
        
        % ----- EXPERIENCE -----
        \vspace{0.1cm}
        \cvsection{Experiences}
            \cvevent{Stagiaire développeur logiciel }{| Maison Blanche Formation}{Avril 2021 -- Juillet 2021}{Besançon, France}
            \begin{itemize}
                \item Réalisation d'un test de niveau de langue évaluant les 4 compétences principales avec le framework Symfony
            \end{itemize}
            \divider
            
            \cvevent{Opérateur d’atelier }{| HAUCK}{Juillet 2019}{Besançon, France}
            \begin{itemize}
                \item Aide à la préparation des pièces pour traitement thermique
            \end{itemize}
            \divider
            
            \cvevent{Opérateur d’atelier }{| PLAST MOULDING}{Juin 2019}{Besançon, France}
            \begin{itemize}
                \item Assemblage de sous-ensembles pour l’automobile
            \end{itemize}
            \divider
            
            \cvevent{Opérateur magasinier }{| DATC Europe}{Juillet 2017}{Besançon, France}
            \begin{itemize}
                \item Préparateur de commandes, réception de colis de produits finis
                \item Montage de composants mécaniques
            \end{itemize}
            
        % ----- EXPERIENCE -----
        
        % ----- EDUCATION -----
        \cvsection{Formation}
            \cvevent{Ingénieur en Informatique/Électronique }{| Ingéniérie du Logiciel et des Connaissances}{Depuis 2021}{ESIREM, Dijon}
            \divider
            
            \cvevent{DUT Informatique }{}{2019 -- 2021}{IUT de Dijon, Dijon}
            \divider
            
            \cvevent{Anglais intensif }{| Certificat obtenu : Niveau avancée}{2018 -- 2019}{Kaplan Whittier College, Los Angeles}
            \divider
            
            \cvevent{Baccalauréat série S }{| Sciences de la Vie et de la Terre, spécialité Informatique et Sciences du Numérique}{2017 -- 2018}{Lycée Louis Pergaud, Besançon}
            
        % ----- EDUCATION -----
        
        % ----- PROJECTS -----
        \cvsection{Projets}
            %\cvevent{Simulation d'un jeu du type "RISK" }{\cvrepo{| \faGithub D-BASTON}{https://github.com/Orchanyne/D-BASTON}}{}{}
            %Projet réalisé sur un semestre à l’aide du logiciel Unity et du langage C\#. Durant ce projet, mon groupe et moi avons travaillé sur la réalisation d’une interface de jeu, un système de lancer de dés ainsi que sur un mode multijoueurs.
            
            %\divider
            
             \cvevent{Conception et Réalisation d'un "Blind-Test" }{\cvrepo{| \faGithub Blind-Test}{https://github.com/Orchanyne/Blind_test}}{}{}
            Projet réalisé en binôme en 3 mois en utilisant différents frameworks et bibliothèques du Web comme : Node.js, Socket.io et Ajax. De plus, ce projet intègre la gestion d'une base de données.  Ce dernier projet à l'IUT nécessite d'utiliser les différentes compétences acquises tout au long de ces deux années universitaires. 
        % ----- PROJECTS -----
    \end{paracol}
\end{document}
